\documentclass[a4paper]{article}

\usepackage{tabularx}
\bibliographystyle{IEEE}

\renewcommand\refname{Selected publications}
%\usepackage{doublespace}
%\setstretch{1.2}

\usepackage{aeguill}
\usepackage[T1]{fontenc}
\usepackage[utf8]{inputenc}
\usepackage[pdftex]{graphicx}
\usepackage{a4wide}
\usepackage{url}
\usepackage{CV}

\begin{document}

\pagestyle{empty}

%Ueberschrift
\begin{center}
\huge{\textsc{Curriculum Vitae}}
\vspace{0.3cm}

\Large{\textsc{Simon Kågström}}
\end{center}
\vspace{0.5cm}

\section{Address~~~~~~~~~~~~~~~~~~~~~~~~~~~~Contact}
\begin{table}[h]
\begin{tabular}{@{}lll@{}}
  Sankt Mickelsgatan 176~~~~~~~~~~~~~~~~~~~~~~&  Phone (mobile):& +46-709-263844 \\
  129 44 Stockholm & \\
  Sweden & Email: & \texttt{simon.kagstrom@gmail.com} \\
\end{tabular}
\end{table}


\section{Education}

\begin{CV}
\item[2002--2008] Ph.D. in Computer Systems Engineering at Blekinge Institute of Technology,
  Sweden.

  \begin{itemize}
  \item Research on SMP operating system porting and binary translation.
  \item Research member of the Parallel Architecture and Applications for
    Real-Time systems group.
  \end{itemize}

\item[1998--2002] MA of Computer Science, Lund University, Sweden.
  \begin{itemize}
  \item Specialization: Operating systems, compiler construction.
  \end{itemize}

\item[1996--2000] BA of Human Geography, Lund University, Sweden.

\end{CV}

\section{Work experience}

\begin{CV}
\item[2009--] Net Insight AB, Sweden
  \begin{itemize}
    \item Design and implementation of C++17 code in a distributed system, from FPGA 
      register accesses up to the CLI and web support interface
    \item Test-driven development with a heavy focus on unit testing in C++ and system testing
          in Python
    \item Control software for 100Gb Ethernet- and terabit-scale DTM switches
    \item IP encapsulation for video feeds (SMPTE 2110, 2022-6)
    \item Design and implementation DPDK driver of a custom high-performance network interface
    \item Linux kernel driver for the same network interface
  \end{itemize}
At Net Insight, I've become a very experienced C++ programmer and learned to love
test driven development via unit testing.

Keywords: git, C++17, unit testing, Python, DPDK, kernel drivers, Ethernet

\item[2006--2009] Ericsson AB, Sweden

  \begin{itemize}
  \item Adaptation of the Linux kernel for custom Freescale PowerPC-based
    switch boards as well as kernel drivers for specialized hardware
  \item A Linux driver for a custom interrupt controller
  \item A Linux-based root filesystem and basic userland (``distribution'')
  \end{itemize}

At Ericsson, I grew used to writing code for the Linux kernel.

Keywords: C, kernel drivers

\item[Fall 2002--2008] Blekinge Institute of Technology, Sweden

  \begin{itemize}
  \item Implementor of SMP support for the Dicos operating system
    kernel in the Ericsson TSP system on Intel IA-32 hardware.
  \item Implementor of the Cibyl binary translator and programming environment
    for C/C++ support in a J2ME environment, including aggresive optimizations
    and GDB support
  \item Lecturing and assisting undergraduate students in low-level
    programming (MIPS and Intel IA-32 assembly) and game programming courses.
  \end{itemize}

During my PhD studies, I learned to pursue and investigate novel ideas, and
got a good understanding of very low-level development.

Keywords: research, assembly language, operating systems

\item[Summer 2001, spring 2002] University of Karlsruhe, Germany.
  \begin{itemize}
  \item Research student the Systems Architecture group, system-level work
    with the L4Ka::Hazelnut microkernel.
  \item As MSc. thesis, a device driver framework for the Prime multiserver
    operating system was designed and implemented. The device driver framework
    handles virtual to physical address translation through a hierarchy of
    memory servers as well as time-limited memory pinning for DMA.
  \item The Qt/Embedded library was ported to run directly on top of the
    L4Ka::Hazelnut kernel, with a minimal libc implemented and drivers for
    SVGA graphics and terminal input.
  \end{itemize}

Keywords: operating systems, kernel drivers


\item[Summers 2000, 2002] Axis communications, Lund, Sweden.
  \begin{itemize}
  \item Monitoring extensions, bug fixes and test programs for the Linux journalling
    flash file system (JFFS) was implemented.
  \item A presentation tool which displays and selects interesting video feeds
    from multiple network cameras in realtime was implemented.
  \end{itemize}

\item[Other] Participation in open source projects (https://github.com/SimonKagstrom)

During the years, I've written and contributed to a lot of open source software, most
of which is available on my github account. I make programs because it's fun, and making
them open source allows other people to use and contribute to the programs. Some highlights
are:

  \begin{itemize}
    \item Implementor of Kcov, a code coverage tool similar to gcov
      which instruments binaries and tests coverage without compiler
      options.
    \item Designer and imlpementor of the Cibyl programming environment, a
      binary translator MIPS-Java bytecode which allows C/C++ programs
      to run on J2ME-capable mobile phones. Using this, I was the first person
      on Earth to finish Space Quest on a mobile phone.
    \item Implementor of EmilPRO and Dissy, two versions of graphical disassemblers for
      multiple architectures in Qt and GTK.
  \end{itemize}
\end{CV}
%\section{References}
%
%\noindent These persons are familiar with my professional qualifications and my character:

%\begin{table}[h]
%\begin{tabular}{@{}ll@{}}
%\textbf{Prof. Lars Lundberg}~~~~~~~~~~~~~~~~~~~~~~~~~~~~~~~~~& \textbf{Volkmar Uhlig} \\
%PhD. Thesis supervisor  & MSc. Thesis supervisor \\
%P.O. Box 520 & Luisenstrasse 2\\
%372 25 Ronneby & 76137 Karlsruhe \\
%Sweden & Germany \\
%Email: \texttt{llu@bth.se} & Email: \texttt{volkmar@ira.uka.de}\\
%Phone: +46-457-385833 & Phone: +49-172-3528285
%\end{tabular}
%\end{table}

\bibliography{bib}

\nocite{kagstrom08phd, kagstrom05experiences, kagstrom05appkern,
  kagstrom07cibyl}
\noindent


\end{document}
