\documentclass[a4paper]{article}

\usepackage{tabularx}
\bibliographystyle{IEEE}

\renewcommand\refname{Selected publications}
%\usepackage{doublespace}
%\setstretch{1.2}

\usepackage{aeguill}
\usepackage[T1]{fontenc}
\usepackage[utf8]{inputenc}
\usepackage[pdftex]{graphicx}
\usepackage{a4wide}
\usepackage{url}
\usepackage{CV}

\begin{document}

\pagestyle{empty}

%Ueberschrift
\begin{center}
\huge{\textsc{Curriculum Vitae}}
\vspace{0.3cm}

\Large{\textsc{Simon Kågström}}
\end{center}
\vspace{0.5cm}

\section{Address~~~~~~~~~~~~~~~~~~~~~~~~~~~~Contact}
\begin{table}[h]
\begin{tabular}{@{}lll@{}}
  Sankt Mickelsgatan 176~~~~~~~~~~~~~~~~~~~~~~&  Phone (mobile):& +46-709-263844 \\
  129 44 Stockholm & \\
  Sweden & Email: & \texttt{simon.kagstrom@gmail.com} \\
  &Web: & \url{https://github.com/SimonKagstrom} \\
\end{tabular}
\end{table}


\section{About me}

\begin{CV}
\item
I'm an experienced C++ software developer that trives in cross-functional teams. I find it easy to
collaborate with both other software developers and other disciplines, such as electronics and
FPGA developers, testers and support staff. In many of the projects I have participated in, I
have taken a coordinating role and I’ve also been deeply involved in the software design work.
I love working with test-driven development, and have a strong focus on modern C++ (C++23 as
of now).

Most of my spare time is spent with my wife and two children, and their activities. Apart from
that, I'm also a bicycle geek, and enjoy both commuting by bike and taking apart and repairing
my bicycles. I also enroll in various projects, often combining software and hardware, and have
made a GPS plotter for my boat, an e-paper-based energy monitor as well as a number of pure
software projects.
\end{CV}

\section{Education}

\begin{CV}
\item[2002--2008] Ph.D. in Computer Systems Engineering at Blekinge Institute of Technology,
  Sweden.

  \begin{itemize}
  \item Research on SMP operating system porting and binary translation.
  \item Research member of the Parallel Architecture and Applications for
    Real-Time systems group.
  \end{itemize}

\item[1998--2002] MA of Computer Science, Lund University, Sweden.
  \begin{itemize}
  \item Specialization: Operating systems, compiler construction.
  \end{itemize}

\item[1996--2000] BA of Human Geography, Lund University, Sweden.

\end{CV}

\section{Participation in open source projects}

\begin{CV}

\item[1990s--today]

As a hobby and sometimes for my day-to-day work, I have started and contributed to a
large number of open source projects, most of which are on my github account
(\url{https://github.com/SimonKagstrom}). Some of what I think are highlights are kcov,
maelir, cibyl and dissy / emilpro.

Maelir is a ESP32-based speedometer-style GPS plotter for my classic Norwegian fiberglass
boat, implemented in C++23. Kcov is a code coverage tool which is built around breakpoints
and therefore can be used on non-instrumented binaries. It also supports multiple output
formats and integration with other tools. 

Further back, I wrote Cibyl, which is a binary translator and programming environment
which translates MIPS binaries into Java bytecode to allow execution of a C/C++ program
on a J2ME phone. Via Cibyl, in this way I was probably the first in the world to play
Space Quest on a mobile phone. Dissy and Emilpro are two graphical disassemblers, which
facilitate navigation and address/instruction searching when debugging at a really
low level.

\textbf{Keywords}: C, C++23, Python, git, github, ESP32, LVGL, embedded systems
\end{CV}

\section{Work experience}

\begin{CV}
\item[2022--2025] Consultant at Profoto AB, Sweden

During the consultancy assignment at Profoto, I have worked with professional studio lightning
and flashes for photography. The products are based on STM32 MCUs, and I have worked on C++23
code for target as well as in extensive unit tests (Doctest and Trompeloeil) and a Qt6-based
simulator.

I have been deeply involved in the design of the software platform used, and have implemented
both hardware drivers, internal communication abstractions and the unit test environment. In
addition, I have implemented much of the Python-based automated regression suite targeting
both hardware and the simulator.

The code for the target products involve e.g., display drivers, low-level UART handling,
synchronization between IRQ and task context and the task execution environment. Higher up,
I have implemented the DMX and RDM protocol handling, as well as parts of the user
interface, and communication with radio modules.

\item[2009--2022] Net Insight AB, Sweden
  
At Net Insight, I have worked with media transport on the company's proprietary hardware
platform. During my years at Net Insight, I have implemented code in a highly asynchronous
distributed system in C++17, all the way from controlling the hardware to the web interface.
I have worked with test-driven development using unit tests with Catch2! and the Trompeloeil
frameworks for virtually all implementation. In some of the projects I have also implemented
drivers for high-performance network interfaces, both for DPDK and the Linux kernel.

During the last years at Net Insight, I worked with IP video in a broadcast environment,
mainly SMPTE 2022-6 and SMPTE 2110. The platform is built around a modular Ethernet switch,
and my work spans both control software for the FPGA dataplane and configuration on a higher
level.

My contribution to Net Insight was, in addition to the implementation of C ++ code, also the
design of how the software (and sometimes the FPGA interfaces) should work and be structured.
I have also been a scrum master for a long time in the team I worked in.

\textbf{Keywords}: Git, C++17, Unit testing, Python, DPDK, Kernel drivers, Ethernet, Switching, PlantUML

\item[2006--2009] Ericsson AB, Sweden
At Ericsson, I worked in two departments. First, I worked on the multiprocessor adaptations
of the DICOS operating system on which parts of my dissertation were based. In connection
with the department moving to Budapest in 2007, I spent the summer in Hungary transferring
system knowledge to the Hungarian department via pair programming.

From 2007 I worked in a department that made Ethernet switches that were used internally as
building blocks for other applications. In this role, I implemented Linux support for the
PowerPC-based system one of the switches (CMXB) used, and also wrote drivers for the
specialized hardware, including an interrupt controller. For the CMXB, I also built the root
file system with the support needed for the userland applications, based on a Wind River
Linux distribution.

My contribution to Ericsson was mainly as an expert on the Linux kernel.

\textbf{Keywords}: C, kernel drivers

\item[Fall 2002--2008] Blekinge Institute of Technology, Sweden

  \begin{itemize}
  \item Implementor of SMP support for the Dicos operating system
    kernel in the Ericsson TSP system on Intel IA-32 hardware.
  \item Implementor of the Cibyl binary translator and programming environment
    for C/C++ support in a J2ME environment, including aggresive optimizations
    and GDB support
  \item Lecturing and assisting undergraduate students in low-level
    programming (MIPS and Intel IA-32 assembly) and game programming courses.
  \end{itemize}

During my PhD studies, I learned to pursue and investigate novel ideas, and
got a good understanding of very low-level development.

\textbf{Keywords}: research, assembly language, operating systems

\item[Summer 2001, spring 2002] University of Karlsruhe, Germany.
  \begin{itemize}
  \item Research student the Systems Architecture group, system-level work
    with the L4Ka::Hazelnut microkernel.
  \item As MSc. thesis, a device driver framework for the Prime multiserver
    operating system was designed and implemented. The device driver framework
    handles virtual to physical address translation through a hierarchy of
    memory servers as well as time-limited memory pinning for DMA.
  \item The Qt/Embedded library was ported to run directly on top of the
    L4Ka::Hazelnut kernel, with a minimal libc implemented and drivers for
    SVGA graphics and terminal input.
  \end{itemize}

\textbf{Keywords}: operating systems, kernel drivers


\item[Summers 2000, 2002] Axis communications, Lund, Sweden.
  \begin{itemize}
  \item Monitoring extensions, bug fixes and test programs for the Linux journalling
    flash file system (JFFS) was implemented.
  \item A presentation tool which displays and selects interesting video feeds
    from multiple network cameras in realtime was implemented.
  \end{itemize}

\end{CV}
%\section{References}
%
%\noindent These persons are familiar with my professional qualifications and my character:

%\begin{table}[h]
%\begin{tabular}{@{}ll@{}}
%\textbf{Prof. Lars Lundberg}~~~~~~~~~~~~~~~~~~~~~~~~~~~~~~~~~& \textbf{Volkmar Uhlig} \\
%PhD. Thesis supervisor  & MSc. Thesis supervisor \\
%P.O. Box 520 & Luisenstrasse 2\\
%372 25 Ronneby & 76137 Karlsruhe \\
%Sweden & Germany \\
%Email: \texttt{llu@bth.se} & Email: \texttt{volkmar@ira.uka.de}\\
%Phone: +46-457-385833 & Phone: +49-172-3528285
%\end{tabular}
%\end{table}

\bibliography{bib}

\nocite{kagstrom08phd, kagstrom05experiences, kagstrom05appkern,
  kagstrom07cibyl}
\noindent


\end{document}
